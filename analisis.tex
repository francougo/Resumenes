\documentclass[14pt, letterpaper]{article}
\usepackage[utf8]{inputenc}

\title{Analisis de Sistemas}
\date{Ultima modificacion: \today}

\begin{document}

\maketitle

\textbf{Definicion de Sistema de informacion}\\
Conjunto de elementos o componentes interrelacionados para
capturar, procesar, almacenar, y distribuir la informacion
en un sistema para permitir la toma de decisiones, la gestion
del sistema, y el aprendizaje colectivo.
\textbf{Caracteristicas no funcionales del software}\\
Atributos no directamente relacionados con lo que el
software hace. Estos se definen como atributos de calidad.
Los principales atributos son:\\
\begin{itemize}
        \item Mantenibilidad:
                Debe ser posible que el software
                evolucione y mantenga sus especificaciones
        \item Confiabilidad:
                Debe funcionar sin fallas (en un entorno
                determinado y durante un periodo 
                especifico de tiempo.
        \item Eficiencia:
                No debe desperdiciar los recursos del sistema.
        \item Usabilidad: 
                Debe contar con una interfaz de usuario y 
                una adecuada documentacion.
\end{itemize}
Dependiendo del tipo de software, distintos atributos
tendran mayor relevancia.

\textbf{Categorias de software}\\
\begin{itemize}
        \item Productos genericos, se producen por una 
                organizacion de desarrollo y se venden en el
                mercado a cualquier cliente que desee comprarlo.
        \item Productos hechos a medida, sistemas destinados para
                un cliente en particular.
\end{itemize}

Existen 4 etapas comunes a todos los procesos de ingenieria
de software:
\begin{itemize}
        \item Especificacion del software, se define el 
                software que se producira y sus restricciones.
        \item Desarrollo de software, se disena y programa
        \item Validacion, se verifica que el software es lo que
                el cliente requiere.
        \item Evolucion, se modifica para reflejar los 
                requerimientos cambiantes del cliente/mercado.
\end{itemize}

\textbf{Fundamentos de la ingenieria de software aplicados a
todos los tipos de sistemas de software:}\\
\begin{itemize}
        \item Se debe utilizar un proceso de desarrollo 
                administrado y comprendido, se debe planear el
                proceso, que se producira y cuanto tiempo tomara
                completarlo.
        \item La confiabilidad y el desempeno son importantes en
                todos los casos.
        \item Comprension y gestion en la especificacion y los
                requerimientos del sistema.
        \item Efectividad de los recursos existentes, se debe 
                reutilizar el software donde sea adecuado.
\end{itemize}


\textbf{Herramientas}\\
Sirven para facilitar el trabajo del ingeniero en software
y se clasifican en upper CASE y lower CASE segun si se
utilizan para el desarrollo en alto o bajo nivel de 
abstraccion respectivamente.

\textbf{Metodos}\\
Indican como construir el software.
Todo metodo tiene:\\
\begin{itemize}
        \item Descripcion del modelo del sistema
        \item Reglas
        \item Recomendaciones
        \item Guias del proceso
\end{itemize}
\textbf{Procesos}\\
Son secuencias de actividades para la produccion de software.
Definen la secuencia en la que se aplican los metodos, las 
entregas requeridas, controles (para asegurar calidad) y 
directrices (para los gestores de software).\\
\\
Las principales actividades son las especificaciones,
el desarrollo, la validacion y la evolucion del software.

Caracteristicas del proceso de software:\\
\begin{itemize}
        \item Visibilidad
        \item etc (completar)
\end{itemize}



\end{document}







