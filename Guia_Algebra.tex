\documentclass[14pt, letterpaper]{article}
\usepackage[utf8]{inputenc}
\usepackage{amssymb}

\title{Guia de resolucion de ejercicios de Algebra lineal}
\date{Ultima modificacion: \today}

\begin{document}

\maketitle

\textbf{Programacion lineal}\\
Para resolver ejercicios de programacion lineal primero se deben
plantear las inecuaciones segun los datos que presente el 
ejercicio. Luego se plantea la funcion objetivo, que va a ser
la funcion a maximizar o a minimizar y debe ser de la forma
$ax + by = k$\\
Donde a y b se obtienen del enunciado y las variables corresponden a las cantidades que necesitamos encontrar.
Posteriormente se procede a graficar cada inecuacion en el plano
coordenado, muchas veces es util plantear inecuaciones para 
limitar los posibles resultados a los numeros positivos mediante
$x \geq 0$ y $y \geq 0$ por ejemplo. Son muy frecuentes dichas 
inecuaciones ya que no tiene sentido trabajar con cantidades 
negativas por ejemplo.
El siguiente paso es plantear la funcion objetivo con $k=0$.\\
Si se intenta maximizar la funcion, se debe buscar un valor de k mayor a la zona factible (zona formada por la interseccion de
las inecuaciones) e ir disminuyendolo hasta el primer punto en 
donde la interseccion de la recta de la funcion objetivo y la 
zona factible exista, puede resultar en un unico punto, caso en 
el que esa es la unica solucion, o en una recta, caso en el que
cualquier punto que pertenezca a ella es solucion al problema.\\
Si se intenta minimizar la funcion objetivo, los pasos son los 
mismos pero comenzando con un valor de k inferior a la zona 
factible e incrementandolo.
\clearpage

\textbf{Matrices}\\
Algunas propiedades utiles sobre las operaciones de matrices.\\
Sean A, B y C tres matrices de m x n tal que los siguientes productos
y sumas estan definidos y sean $\alpha$, $\beta$ dos escalares:\\
\begin{itemize}
        \item $A+0 = A$
        \item $0A = 0$
        \item $A+B = B+A$
        \item $(A+B)+C = A+(B+C)$
        \item $\alpha(A+B) = \alpha A + \alpha B$
        \item $(\alpha + \beta)A = \alpha A + \beta A$
        \item $1A = A$
        \item $A(BC) = (AB)C$
        \item $A(B+C) = AB + AC$ y $(A+B)C = AC + BC$
        \item $AB \neq BA$
        \item $(AB)^{-1} = B^{-1} A^{-1}$
        \item $(AB)^T = B^T A^T$
        \item $(A+B)^T = A^T + B^T$
        \item $(A^T)^{-1} = (A^{-1})^T$
\end{itemize}
\clearpage

\textbf{Determinantes}\\
Algunas propiedades de los determinantes:\\
\begin{itemize}
        \item $det(A) = det(A^T)$
        \item Si A es una matriz triangular, $det(A) = $ producto de los elementos
                de la diagonal principal.
        \item $det(AB) = det(A) det(B)$
        \item Si A tiene una fila o columna de ceros $det(A) = 0$
        \item $det(\alpha A) = \alpha ^n det(A)$ donde n es el orden de A.
        \item Si se intercambian dos filas o columnas de A, $det(A)$ cambia su signo.
        \item Si dos filas o columnas de A son proporcionales $det(A) = 0$
        \item Si a una fila o columna se le suma un multiplo de otra fila o columna
                $det(A)$ no cambia.
        \item $det(A^{-1}) = 1/det(A)$
\end{itemize}
\clearpage

\textbf{Vectores}\\
Propiedades y formulas utiles de operaciones de vectores:\\

\textbf{Angulos directores}\\
$x = |\vec v| cos(\alpha)$\\
$y = |\vec v| cos(\beta)$\\
$z = |\vec v| cos(\beta)$\\
\\
$cos^2(\alpha) + cos^2(\beta) + cos^2(\gamma) = 1$\\
Propiedades:\\
\begin{itemize}
        \item $|\alpha \vec v| = |\alpha||\vec v|$
        \item $\vec v.\vec v = |\vec v|^2$
\end{itemize}

Angulo entre dos vectores\\
$cos(\theta)= \frac{\vec{v} . \vec{u}}{|\vec{v}||\vec{u}|}$\\

Paralelismo\\
$\vec u \parallel \vec v \iff \vec u = k \vec v$\\

Perpendiculares
$\vec u \perp \vec v \iff \vec v \vec u = 0$\\

Proyeccion y modulo de la proyeccion\\
$Proy_{\vec v}\vec u = \frac{\vec u \vec v}{|\vec v|^2}\vec v$\\

$|Proy_{\vec v}\vec u| = \frac{|\vec u \vec v|}{|\vec v|}$\\

Distancia de un punto a una recta\\
$dist((x_1;y_1);Recta) = \frac{|ax_1 + by_1 + c|}{\sqrt{a^2+b^2}}$\\

Algunas propiedades del producto cruz y el triple producto escalar\\
\begin{itemize}
        \item $\vec u \times \vec v = -(\vec v \times \vec u)$
        \item $(\alpha \vec u) \times \vec v = \vec u \times (\alpha \vec v) = \alpha (\vec u \times \vec v)$
        \item $\vec u \times (\vec v + \vec w) = \vec u \times \vec v + \vec u \times \vec w$
        \item $\vec u \parallel \vec v \Rightarrow \vec u \times \vec v = \vec 0$
        \item $(\vec u \times \vec v) . \vec w = \vec u (\vec v \times \vec w)$
        \item $\vec u (\vec u \times \vec v) = \vec v (\vec u \times \vec v) = 0$
        \item $|\vec u \times \vec v| = |\vec u||\vec v|sen(\theta) =$ Area del paralelogramo.
        \item $|\vec v(\vec u \times \vec w)| = 0 \Rightarrow \vec u, \vec v, \vec w$ coplanares.
        \item El modulo del triple producto escalar es igual al volumen del paralelepipedo.
\end{itemize}
\clearpage

\textbf{Planos}\\
Propiedades:\\
\begin{itemize}
        \item $\pi_1 \parallel \pi_2 \iff \vec n_1 \parallel \vec n_2$
        \item $\pi_1 \perp \pi_2 \iff \vec n_1 \perp \vec n_2$
        \item El angulo entre dos planos es igual al angulo entre sus vectores normales.
\end{itemize}

Distancia de un punto a un plano\\
$dist((x_0;y_0;z_0);\pi) = \frac{|ax_0 + by_0 + cz_0 + d|}{|\vec n|}$ donde a, b y c son la componentes de $\vec n$.\\

Ecuacion simetrica de una recta\\
$\frac{x - x_0}{u_1} = \frac{y - y_0}{u_2} = \frac{z - z_0}{u_3}$ donde las componentes sub cero corresponden a
un punto en la recta y los denominadores son las componentes del vector director.\\

\textit{Se pueden obtener facilmente los planos proyectantes de una recta tomando solo dos de los 
tres terminos de su ecuacion simetrica, es decir, si se busca el plano proyectante en z, en otras palabras
el plano perpendicular al plano xy, se deben tomar los terminos x e y para despejar de ellos la ecuacion del
plano proyectante.}\\

\textit{Para parametrizar una recta con ecuacion simetrica solo se debe igualar cada termino a el parametro "t"
y despejar la x, y o z segun corresponda.}\\

Angulo entre una recta y un plano\\
$sen(\theta) = \frac{\vec n \vec u}{|\vec n||\vec u|}$\\

Distancia entre dos rectas\\
$dist = \frac{|\vec{P_1 P_2} (\vec u_1 \times \vec u_2)|}{|\vec u_1 \times \vec u_2|}$ donde $P_i$ son puntos pertenecientes a las rectas
y $\vec u_i$ son los vectores directores.\\
En caso de que las rectas se corten o sean paralelas su distancia es cero en el primer caso y en el segundo caso
se puede obtener de manera sencilla tomando un punto cualquiera en una recta y utilizando la formula
de distancia entre punto y recta.\\

Trazas\\
Para obtener las trazas de un plano, se debe plantear un sistema con la ecuacion 
del plano y la ecuacion del plano que se quiera obtener la traza (estos pueden ser
el xy, el xz y el yz) ya que las trazas de un plano son las intersecciones de este
con los planos mencionados. Cabe destacar que un plano en $R^3$ puede tener 2 o 3 trazas.\\

\clearpage

\textbf{Espacios y subespacios vectoriales}\\
Todo V tiene al menos dos subespacios vectoriales:
$S = \{0\}$\\
$S = \{V\}$\\
Si posee otros, estos se llaman subespacios propios.\\

Un subconjunto no vacio H de un V es un subespacio de V si:\\
\begin{itemize}
        \item $x, y \in H \Rightarrow x + y \in H$
        \item $x \in H \Rightarrow \alpha x \in H \forall \alpha$ escalar.
\end{itemize}

\underline{Teoremas y propiedades}\\
\begin{itemize}
        \item Si V tiene dos subespacios vectoriales $H_1$ y $H_2$, $H_1 \cap H_2$ es
                otro subespacio vectorial de V.
        \item El conjunto solucion de cualquier sistema de ecuaciones homogeneo es SEV de $\mathbb{R}^n$.
\end{itemize}

\underline{Conjunto generador}\\
$\vec v_1 , \vec v_2 , \vec v_n$ son el conjunto generador (gen()) de un V si y solo si
pertenecen a dicho V y todo $\vec v \in V$ puede escribirse como combinacion lineal de
dichos vectores.\\

\underline{Dependencia e independencia lineal}\\*
$\vec v_1 , \vec v_2 , \vec v_n \in V$ son linealmente dependientes (LD) si el vector
nulo se puede escribir como combinacion lineal de los n vectores donde al menos un escalar no es nulo.\\*
\begin{itemize}
        \item $\vec v , \vec u$ son LD $\iff \vec v = k \vec u$
\end{itemize}

\textit{Se puede plantear un sistema de ecuaciones para determinar si un conjunto de vectores es 
linealmente independiente de la siguiente forma:\\*}
$\left(\begin{array}{ccc}
        u_1 & v_1 & w_1 \\
        u_2 & v_2 & w_2 \\
        u_3 & v_3 & w_3
\end{array}|
\begin{array}{c}
        0 \\
        0 \\
        0
\end{array}\right)\\*
$        
        
\begin{itemize}
        \item SCI $\Rightarrow$ LD
        \item SCD $\Rightarrow$ LI
\end{itemize}

\textit{Cabe destacar que no puede resultar en un sistema incompatible ya que es homogeneo}\\*

\textit{Mas teoremas y propiedades}\\*
\begin{itemize}
        \item Tres vectores en $\mathbb{R}^3$ son LD $\iff$ son coplanares.
        \item Si el numero de vectores es mayor al numero de componentes de cada
                vector entonces son LD.
        \item Cualquier conjunto de n vectores LI en $\mathbb{R}^n$ genera a $\mathbb{R}^n$.
\end{itemize}

\textbf{Bases}\\*
Un conjunto finito de vectores es una base para V si:\\*
\begin{itemize}
        \item Los n vectores son LI
        \item Los n vectores generan a V
\end{itemize}

\textit{Para encontrar bases de un subespacio se busca el vector generico del mismo, es decir
un vector en el que se exprese la relacion (si la hay) entre sus componentes y luego se 
descompone dicho vector como combinacion lineal de vectores. El resultado debe contener
tantos vectores como variables distintas tenga el vector generico y este es la base del SEV.}\\

Un vector de coordenadas esta formado por los escalares por los que se debe multiplicar a la base
de un SEV para obtener el vector representado, por lo tanto debe ser empleado solo para bases ordenadas.\\*

\textit{Espacio vectorial con producto interno}\\*
\begin{itemize}
        \item Norma es equivalente a modulo y ortogonal a perpendicular.
        \item Un conjunto es ortogonal si cada vector es ortogonal con los demas
                vectores del conjunto.
        \item Un conjunto es ortonormal si es ortogonl y ademas, para todo $\vec v \in \mathbb{R}^n :
                \vec v . \vec v = 1$ es decir posee norma 1.
        \item Una base puede ser tanto ortogonal como ortonormal.
\end{itemize}

\textit{Para cambiar un vector a otra base se debe expresar dicho vector como combinacion
lineal de la base objetivo y resolver el sistema de ecuaciones.}\\*
\clearpage

\textbf{Transformaciones lineales}\\*
Una funcion que asigna a cada vector $\in V$ un unico vector $T \vec v \in W$ y que
para cada $\vec u , \vec v \in V ; \alpha \in \mathbb{R}$:\\*
\begin{itemize}
        \item $T(\vec u + \vec v) = T \vec u + T \vec v$
        \item $T(\alpha \vec v) = \alpha T \vec v$
\end{itemize}

\textit{Para verificar que una transformacion es lineal a partir de su forma analitica, se 
deben plantear ambas condiciones anteriores con el vector generico y comprobar que se
satisfacen.}\\*

Propiedades\\*
\begin{itemize}
        \item $T(\vec 0) = \vec 0$
        \item $T(\vec u - \vec v) = T \vec u - T \vec v$
        \item $T:\mathbb{R}^n \rightarrow \mathbb{R}^m$ es TL $\iff \exists A_{m \times n} / T \vec v = A \vec v
                \forall \vec v \in \mathbb{R}^n$
        \item $T(\alpha_1 \vec v_1 + ... + \alpha_n \vec v_n) = \alpha_1 T \vec v_1 + ... + \alpha_n T \vec v_n$
                es decir que la transformacion de una combinacion lineal es igual a la combinacion lineal de
                los transformados.
\end{itemize}

\textit{Por la ultima propiedad presentada se puede calcular el efecto de una transformacion lineal
sobre cualquier vector en base A si se conocen los transformados de los vectores de la base A, ya que
todo vector en base A se puede escribir como combinacion lineal de la base A.}\\*

\textit{Para obtener la matriz asociada a una transformacion lineal basta con colocar como columnas
a los transformados de la base estandar. Pero si se pide hayar la matriz de un transformado $M_{BC}(T)$\\*
donde B es una base en el conjunto de partida y C una en el conjunto de llegada, se deben transformar
en primer lugar los vectores de la base B segun la TL y luego convertir los resultantes a base C.}\\*

\underline{Nucleo e imagen de una transformacion}\\*
$nu(T) = \{\vec v \in V: T \vec v = 0\}$\\*
$im(T) = \{\vec w \in W: \vec w = T \vec v$ para algun $\vec v \in V\}$\\*

\textit{Tanto el nucleo como la imagen de una TL son espacios vectoriales y se obtienen de
resolver un sistema de ecuaciones igualando el vector generico $T(\vec v)$ a el vector
generico de $\mathbb{R}^n$ o al vector nulo, respectivamente. En el caso de la imagen
se busca la condicion para que el sistema sea compatible y en el nucleo solo se resuelve
ya que es un sistema homogeneo.}\\*

El rango y la nulidad son las dimensiones de la imagen y el nucleo respectivamente.\\*
\clearpage

\textbf{Eigenvalores y eigenvectores}\\*
$\lambda$ es un valor caracteristico de una matriz asociada a una TL $A \iff det(A- \lambda I) = 0$\\*

Los eigenvectores se obtienen de resolver el sistema de ecuaciones $(A- \lambda I)\vec V = \vec 0$\\*
Donde V es el vector generico de $\mathbb{R}^n$. La solucion al sistema sera un SEV asociado al 
eigenvalor correspondiente, la base de este subespacio esta conformada por los eigenvectores.

\underline{Teoremas y propiedades}\\*
\begin{itemize}
        \item Vectores caracteristicos correspondientes a eigenvalores distintos son LI
        \item Los eigen valores de una M triangular son los elementos de su diagonal principal
        \item Multiplicidad Geometrica $= dim(E_{\lambda})$
        \item $A_{n \times n}$ tiene n vectores LI $\iff \forall \lambda_i : MG(\lambda_i) = MA(\lambda_i)$
\end{itemize}

Dos matrices M y A son semejantes si $M = C^{-1}AC$.\\*

\underline{Propiedades}\\*
\begin{itemize}
        \item Si dos matrices son semejantes, tienen el mismo polinomio caracteristicos, los mismos eigenvalores
                y la misma traza
        \item Se puede cambiar de base una TL buscando una matriz semejante siendo la matriz C la matriz que contiene
                a los vectores de la base objetivo
        \item Si A es semejante a B y A es diagonal, entonces B es diagonalizable
        \item $M_{n \times n}$ es diagonalizable si y solo si tiene n vectores caracteristicos LI
        \item Sea A diagonalizable y D una matriz similar diagonal:\\*
                $A^n = CD^n C^{-1}$\\*
                Donde las columnas de C son los eigenvectores de A ordenados segun la diagonal principal de D.
\end{itemize}
\clearpage

\textbf{Conicas}\\*
\underline{Ecuacion General de segundo grado (incompleta)}\\*
$Ax^2 + Cy^2 + Dx + Ey + F = 0$\\*
\underline{Circunferencia}\\*
$(x-k)^2 + (y-h)^2 = r^2$\\*
Donde $(k;h)$ es el centro y r es el radio.\\*

Condiciones\\*
\begin{itemize}
        \item $A = C \neq 0$
        \item $M > 0$ donde $M = \frac{-4AF+D^2+E^2}{4A^2}$
\end{itemize}

\underline{Parabola}\\*
El eje focal de la parabola siempre es paralelo al eje correspondiente al termino lineal\\*
$(x-h)^2 = 2p(y-k)$\\*
Donde $(h;k)$ es el vertice y p es la distancia del foco a la directriz.\\*

Condiciones\\*
\begin{itemize}
        \item $C = 0; A \neq 0; E \neq 0$
        \item o $C \neq 0; A = 0; D \neq 0$
\end{itemize}

\underline{Elipse}\\*
$\frac{(x-h)^2}{a^2} + \frac{(y-h)^2}{b^2} = 1$\\*
Donde $(h;k)$ es el centro, a es la distancia entre un foco y un vertice principal, b es la
distancia entre el centro y un vertice principal y c es la distancia entre el centro y un foco.\\*
\begin{itemize}
        \item $a^2 = b^2 + c^2$
        \item Lado recto $= \frac{2b^2}{a}$
        \item Exentricidad $= \frac{c}{a}$
        \item Condiciones: el signo de A, C y M es igual, siendo M el termino producido al completar
                cuadrados.
\end{itemize}

\underline{Hiperbola}\\*
$\frac{(x-h)^2}{a^2} - \frac{(y-k)^2}{b^2} = 1$\\*
Cualquier termino puede ser negativo, pero solo debe ser uno y $a^2$ siempre divide al positivo.
Si x es positivo el eje focal es horizontal, si y es positivo el eje focal es vertical.\\*
a es la distancia entre el centro y un vertice, b entre el vertice y la asintota, perpendicular al EF
y c entre el centro y un foco. Siempre se cumple que $c^2 = a^2 + b^2$\\*

Ecuaciones asintotas\\(
EF horizontal\\*
$y-k = +- \frac{b}{a}(x-h)$\\*
EF vertical es igual pero la pendiente es $\frac{a}{b}$.\\*
\end{document}
















