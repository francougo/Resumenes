\documentclass[14pt, letterpaper]{article}
\usepackage[utf8]{inputenc}

\title{Resumen y guia para resolver ejercicios de Analisis Matematico}
\date{Ultima modificacion: \today}

\begin{document}
\maketitle

\centerline{\textbf{Funciones y modelos}}\*
\underline{Transformaciones de funciones}\\*
Desplazamientos:\\*
\begin{itemize}
        \item $y = f(x) + c$, desplazamiento de $f(x)$ c unidades hacia arriba.
        \item $y = f(x - c)$, desplazamiento de $f(x)$ c unidades hacia la derecha.
\end{itemize}
Alargamientos y reflexiones:\\*
Suponiendo que $c > 1$\\*
\begin{itemize}
        \item $y = c f(x)$, alargamiento de $y = f(x)$ verticalmente.
        \item $y = (1/c)f(x)$, compresion de $y=f(x)$ verticalmente.
        \item $y = f(cx)$, compresion de $y=f(x)$ horizontalmente.
        \item $y = f(x/c)$, alargamiento de $y=f(x)$ horizontalmente.
        \item $y = -f(x)$, reflexion respecto del eje x.
        \item $y = f(-x)$, reflexion respecto del eje y.
\end{itemize}

\underline{Conceptos:}\\*
\begin{itemize}
        \item \underline{Funcion par}: $f(x) = f(-x)$
        \item \underline{Funcion impar}: $f(-x) = -f(x)$
        \item \underline{Supremo}: La menor de todas las cotas superiores de una funcion (acotada).
        \item \underline{Infimo}: La mayor de todas las cotas inferiores de una funcion.
        \item Si el supremo pertence a la funcion se denomina maximo absoluto y si el 
                infimo pertenece a la funcion es el minimo absoluto.
        \item \underline{Funcion inyectiva}: Funcion que posee un unico valor de $x$ para cada valor de $y$\\*
                Es decir, $x_1 \neq x_2 \Rightarrow f(x_1) \neq f(x_2)$
        \item \underline{Funcion sobreyectiva}: Funcion en la que todo elemento del codominio es imagen de
                algun elemento del dominio.\\*
                Es decir, $f:A \rightarrow B$ sobreyectiva $\iff \forall y \in B, \exists x \in
                A / y = f(x)$
        \item \underline{Funcion biyectiva}: Funcion inyectiva y sobreyectiva.
\end{itemize}
\underline{Propiedades de funciones pares e impares}\\*
\begin{itemize}
        \item La suma o diferencia de dos funciones pares es otra funcion par.
        \item La suma o diferencia de dos funciones impares es otra funcion impar.
        \item El producto o cociente entre una funcion par y una impar es una funcion impar.
        \item El producto o cociente entre dos funciones pares o entre dos funciones impares es una
                funcion par.
\end{itemize}

\underline{Funcion periodica}\\*
$f(x)$ es periodica $\iff \exists p > 0 / f(x \pm p), \forall x \in D_f$ p es el periodo de $f(x)$.

\underline{Funcion inversa}\\*
Para que una funcion admita inversa, esta debe ser biyectiva.\\*
Para toda $f(x):A \rightarrow B$ que admita inversa:\\*
\begin{itemize}
        \item $f^{-1}(x)=y \iff f(y) = x$
        \item $f^{-1}(f(x))=x, \forall x \in A$
        \item $f(f^{-1}(x)) = x, \forall x \in B$
\end{itemize}

\textit{Para obtener la inversa de una funcion primero se debe verificar que sea biyectiva, si
no lo es se puede restringir su dominio para alcanzar la condicion. Luego se resuelve la ecuacion
expresando x en terminos de y. Si se requiere expresar a la inversa en funcion de x, se intercambia
y por x.}
\clearpage

\centerline{\textbf{Limites y continuidad}}
\underline{Entorno}\\*
Sean a, h reales, se llama entorno de centro en a y radio h al intervalo abierto $(a-h;a+h) = E(a,h)$.\\*
Es decir, $E(a,h) = {x/a-h<x<a+h}$.\\*
Si se excluye del entorno su centro a, este se denomina entorno reducido.\\*

\underline{Propiedades, condiciones y leyes de los limites}\\*
\begin{itemize}
        \item \(\lim_{x \to a}f(x)=L \iff \lim_{x \to a^-}f(x)=L \and \lim_{x \to a^+}f(x)=L\)
        \item Si \(f(x)=g(x)\) cuando \(x \neq a\), entonces \(\lim_{x \to a}f(x)=\lim_{x \to a}g(x)\)\\
                si el limite existe.
\end{itemize}


\centerline{\textbf{Asintotas}}
\underline{Verticales}\\*
Una recta \(x=a\) es la asintota vertical de \(f(x)\) si se cumple al menos un enunciado:\\*
\[\begin{array}{ccc}
        \lim_{x \to a}f(x)=\infty &\qquad \lim_{x \to a^-}f(x)=\infty &\qquad \lim_{x \to a^+}f(x)=\infty\\
        \lim_{x \to a}f(x)=-\infty &\qquad \lim_{x \to a^-}f(x)=-\infty &\qquad \lim_{x \to a^+}f(x)=-\infty
\end{array}\]
\end{document}

