\documentclass[14pt, letterpaper]{article}
\usepackage[utf8]{inputenc}

\title{Sintaxis y semantica de los lenguajes}
\date{Ultima modificacion: \today}

\begin{document}

\maketitle

\textbf{Simbolo}
Es una entidad indivisible. Pueden estar formados por varias letras.\\

\textbf{Alfabeto}
Conjunto de simbolos finito y no vacio. Se designa con $\Sigma$ o una 
letra mayuscula y un subindice $A_1$\\

\textbf{Cadenas de caracteres}
Con los simbolos de un alfabeto se pueden formar secuencias, tambien 
llamadas cadenas o palabras. Una cadena es una secuencia finita de simbolos
de un determinado alfabeto.\\
Una cadena puede ser vacia, es decir no contener ningun simbolo. Por lo 
tanto pertenece a todos los alfabetos. Se designa con $\epsilon$.\\

\textbf{Longitud de una cadena}
Numero de simbolos de una cadena. Se designa de la siguiente forma $|w|$.\\

\textbf{Concatencion de cadenas}
Sean x e y dos cadenas, xy es su concatenacion.\\
Si $x=a_1a_2...a_i$ y $y=b_1b_2...b_j$ entonces 
$xy=a_1a_2...a_ib_1b_2...b_j$
\begin{itemize}
    \item Es una operacion binaria definida sobre el conjunto universo del alfabeto
    \item El conjunto universo es cerrado bajo la concatenacion
    \item Es asociativa
    \item Elemento identidad = Cadena vacia ($\epsilon$)
    \item No conmutativa
    \item $\forall$ w,v del mismo universo $|wv| = |w|+|v|$
    \item Concatenacion de k copias de $w: w^k = ww...w \} k$ veces
\end{itemize}

\textbf{Potencia de un alfabeto} 
Es el conjunto de cadenas de longitud k tales que todos los simbolos que las forman 
pertenecen al alfabeto.
\begin{itemize}
    \item Potencia 0 $= \epsilon $,
    \item Potencia n $= |\Sigma|^n$
\end{itemize}

\textbf{Universo}
Conjunto de todas las palabras que se pueden definir sobre un alfabeto
Es la union de las infinitas potencias del alfabeto.\\

\underline{Nota:} Cadena vacia ($\epsilon$) $\neq$ Conjunto formado por una cadena 
vacia (\{$\epsilon$\}) $\neq$ Conjunto vacio ($\emptyset$).\\

\textbf{Lenguaje formal}
Conjunto finito o infinito de cadenas de longitud finita, generadas a partir
de un conjunto finito de simbolos llamado alfabeto.\\

\begin{itemize}
    \item Al ser conjuntos se pueden aplicar operaciones de conjuntos.
    \item Sea $\Sigma$ un alfabeto y L un subconjunto del universo $\Sigma^*$, entonces L es un
lenguaje de $\Sigma$ .
    \item Se puede definir a un lenguaje de la siguiente forma:
$L = \{a^n b^n , para\: n>1 \}$
    \item El lenguaje vacio esta incluido en todos los alfabetos
    \item El lenguaje formado por $\epsilon$ esta incluido en todos los alfabetos
    \item \textbf{Concatenacion} $L_1L_2 = \{\omega / \omega = \alpha \beta , \alpha \in L_1,
\beta \in L_2 \}$

\end{itemize}

\textbf{Potencia de un lenguaje}\\
$L^n = \{\epsilon\}, si\: n = 0;\\
     = L.L^{n-1}, si\: n>0$\\

\textbf{Clausura de Kleene}\\
Union de todas las potencias de un lenguaje.\\
$L^* = \bigcup_{i=0} ^ \infty L^i = L^0 \cup L^1 \cup L^i \cup ... \cup L^ \infty$\\

\textbf{Tipos de Gramaticas}\\
Una gramatica se define como $G = \{N, T, S, P\}$\\
donde N es el conjunto de simbolos no terminales, 
T el de los terminales, $S \in N$ el simbolo inicial (el mas general),
y P son las reglas de asignacion para formar palabras de la forma
$\phi A \rho \longrightarrow \phi \omega \rho$ donde $\phi, \rho$ son el contexto.
\begin{itemize}
        \item Tipo 0: Libres de restriccion.
        \item Tipo 1: Dependientes del contexto, $\omega \in (N \cup T)^+$ (No incluye a $\epsilon$).
        \item Tipo 2: Independientes del contexto, $\phi = \rho = \epsilon$.
        \item Tipo 3: Regulares, lineales a la derecha o izquierda. (completar)
        \item Todos los tipos incluyen las restricciones de los tipos que le preceden.
\end{itemize}

\underline{Lineal a derecha e izquierda respectivamente:}\\
$Nu \longrightarrow 0$\\
$Nu \longrightarrow 1$\\
$Nu \longrightarrow B$\\
$B \longrightarrow 0$\\
$B \longrightarrow 1$\\
$B \longrightarrow B0$  -por izquierda\\
$B \longrightarrow 0B$  -por derecha\\

\textbf{Automatas finitos}
Son un modelo util para construir analizadores lexicos.
Son sistemas o componentes con un numero finito de estados,
un estado recuerda la parte significativa de la historia del
sistema.
Sus entradas son palabras del lenguaje que representa.

Un automata tiene tres tipos de estados:
\begin{itemize}
        \item Estado inicial, permite empezar la ejecucion
        \item Estados finales, permite realizar la salida
                de aceptacion
        \item Estados intermedios
\end{itemize}

\textbf{Automatas deterministas}\\
No puede estar en mas de un estado simultaneamente.
Se denota con la quintupla $A = \{\Sigma, Q, f, q_0, F\}$
donde:
\begin{itemize}
        \item $\Sigma$ es el alfabeto de simbolos de entrada
        \item Q es el conjunto finito de estados
        \item $q_0 \in Q$ es el estado inicial
        \item f es la funcion de transicion, toma como argumentos
                un estado de Q y un simbolo de entrada de $\Sigma$
                y devuelve un estado de Q
        \item $F \subseteq Q$ es el conjunto de los estados finales
\end{itemize}
La funcion de transicion "f" es una funcion binaria que 
toma un estado \[\in Q\] y un simbolo \[\in \Sigma\] y
produce un nuevo estado:\\
\[f(q_i,a) = q_j \]\\*

La funcion extendida \[\hat f\] toma un estado y una cadena formada por simbolos del alfabeto del automata y produce un estado. Esta definida de la siguiente forma:\\*
\[\hat f (q, \epsilon) = q\]\\
\[\hat f (q, w) = f (\hat f (q, x), a)\]\\*
Donde a es el ultimo elemento de w y x el resto del cuerpo de w.\\*

Un AFD acepta o no una secuencia de simbolos de entrada segun si
el automata se encuentra en un estado de aceptacion.
Un AFD puede representarse tanto como una tabla de 
transiciones como un grafo o diagrama.

\textbf{Automatas no deterministas}\\
Puede estar en varios estados al mismo tiempo.\\*
Se definen de forma similar a los AFD pero su funcion de transicion, que toma un estado y un simbolo
de entrada, produce un conjunto de estados como salida en vez de un unico estado como en los AFD. 
Este conjunto puede ser vacio.\\*
Por lo tanto tambien posee una funcion de transicion extendida diferente: dicha funcion producira
tambien un conjunto de estados que se obtiene de la siguiente forma:\\*
Sea $\omega = xa$\\*
y sea $\hat f_N(q, x)=\{p_1, p_2, ..., p_k\}$\\*
\[\bigcup_{i=1}^k f_N(p_i, a)=\{r_1, r_2, ..., r_m\} = \hat f_N(q, \omega)\]

\textbf{Automatas finitos con transiciones \(\epsilon\)}\\*
Permite transiciones para la cadena vacia.\\*
Equivalencia con un AFD:\\*
\( AFN-\epsilon\) \(E=(Q_E, \Sigma, f_E, q_0, F_E)\), \(AFD \) \(D=(Q_D, \Sigma, f_D, q_D, F_D)\)\\*
tal que  \(L(D)=L(E)\)\\*

\begin{itemize}
        \item \(Q_D es el conjunto de subconjuntos de Q_E\). Aunque solo los subconjuntos cerrados en 
\(\epsilon\) seran accesibles por el AFD.
        \item \(q_D=CLAUSURA_{\epsilon}(q_0)\)
        \item \(F_D\) incluye todos los conjuntos de estados de E que incluyen al menos
                un estado de aceptacion de E. 
        \item Para cada subconjunto de \(Q_D\) y para cada simbolo de entrada $a$ en $\Sigma$,\\*
                Sea $S = \{p_1, p_2, ..., p_k\}$\\*
                y sea $\bigcup_{i=1}^k f_E(p_i, a) = {r_1, r_2, ..., r_m}$\\*
                $f_D(S, a)=\bigcup_{j=1}^m CLAUSURA_{\epsilon}(r_j)$
\end{itemize}

\textbf{Gramaticas libres de contexto}\\*
Todas las GIC se conforman por:\\*
\begin{itemize}
        \item Un conjunto de simbolos terminales T
        \item Un conjunto de variables o simbolos no terminales N
        \item Un conjunto de reglas de produccion P que definen a cada variable como una composicion de simbolos terminales
                y variables
        \item Un simbolo $S \in N$ que representa a todo el lenguaje
\end{itemize}
\textbf{Derivaciones}\\*
Las derivaciones consisten en, a partir de la variable que representa a todo el lenguaje de una gramatica S,
sustituirla por alguna de sus producciones y al conjunto de variables y simbolos terminales resultante realizarle
la misma accion (para cada variable) hasta llegar a un conjunto de simbolos terminales, en particular interesa
llegar a la cadena que se busca derivar ya que es una forma de demostrar que dicha cadena pertenece al lenguaje.\\*

\textbf{Derivaciones a derecha e izquierda}\\*
Con el fin de ordenar los pasos de una derivacion se puede derivar por ejemplo "a derecha" lo que significa que en 
cada paso se sustituye la variable ubicada mas a la derecha, la derivacion por izquierda es analoga.\\*

\textbf{Lenguaje de una GIC}\\*
El lenguaje de una gramatica libre de contexto se define como el conjunto de cadenas que se pueden obtener como una derivacion
partiendo de la variable inicial de la gramatica.\\*
\textbf{Expresiones regulares}\\*
\begin{itemize}
        \item $\epsilon$ y $\emptyset$ son expresiones regulares que representan a los lenguajes $\{\epsilon\}$ y $\emptyset$
respectivamente.
        \item Si $a$ es un simbolo de entrada, $a$ es una expresion regular que representa al lenguaje $\{a\}$
        \item Una variable \textit{L} representa a cualquier lenguaje
        \item Si E y F son expresiones regulares $E+F$ es una expresion regular y representa a la union $L(E) \cup L(F)$
        \item Si E y F son regex EF es una regex y representa a la concatenacion de lenguajes $L(E).L(F)$
        \item Si E es una regex $E^*$ es una regex que representa a la clausura del lenguaje $(L(E))^*$
        \item Si E es una regex $(E)$ es la misma regex, representa al mismo lenguaje
\end{itemize}
La precedencia de operadores de regex es la siguiente:\\*
Clausura $*$, Concatenacion $.$, Union $+$. Se pueden utilizar parentesis para utilizar otra precedencia.\\*

\textit{Teoria: Los AFD, AFN, $AFN-\epsilon$ y las expresiones regulares representan a los mismos lenguajes.}\\*

Conversion de AFD a regex:\\*
La regex es la union de todas las $R_{1j}^n$ con $n \in [1;k]$ y suponiendo que el estado inicial se etiquete con 1
tal que j sea un estado final.\\*
\end{document}








